	% 代码分析:模块功能、涉及到的类、类关系、数据结构及关键代码等;
	% 任务要求,设计任务要求;
	% 设计:详细的设计方案,相关的数据结构、算法描述,可采用伪代码等形式化描述
	% 实现:修改哪些类、如何修改、为什么修改等;
	% 测试:测试用例,测试结果及结果分析,测试运行界面等;
	% 调试:调试方法,遇到的问题及解决方案等;
	% 结论与展望:完成的主要工作、收获、进一步的工作,建议、体会、心得等;

\section{作业5}
\subsection{作业5.1}
\subsubsection{题目描述}
C风格声明语句文法为 \(G\left\lbrack L\right\rbrack = \left( {\{ L,D,T\} ,\{ ;,,,\text{id,int,double}\} ,P,L}\right)\) ,产生式如下:

\(L \rightarrow L;D\) $\quad$ \(L \rightarrow D\) $\quad$ \(D \rightarrow T \;id\) $\quad$ \(D \rightarrow D,{id}\quad T \rightarrow\) int $\quad$ \(T \rightarrow\) double

\begin{enumerate}
    \item 消除文法产生式的左递归;
    \item 构造所有非终结符号的首符集 First;
    \item 构造所有非终结符号的后继符集 Follow;
    \item 构造 $LL(1)$ 分析表;
    \item 给出句子int \(i\) ; double \(x,y\) 的分析过程。
\end{enumerate}

\subsubsection{解答}

\paragraph{消除文法产生式的左递归} 转为右递归文法:

\(L \rightarrow L;D\) $\quad$ \(L \rightarrow D\) 替换为:

\begin{itemize}
    \item $L \rightarrow  D{L}^{\prime }$
    \item ${L}^{\prime } \rightarrow  ;D{L}^{\prime } \mid  \varepsilon$
\end{itemize}

\(D \rightarrow T \;id\) $\quad$ \(D \rightarrow D,{id}\) 替换为:

\begin{itemize}
    \item $D \rightarrow T\;id\;{D}^{\prime }$
    \item ${D}^{\prime } \rightarrow ,id$ ${D}^{\prime } \mid  \varepsilon$
\end{itemize}

\paragraph{First} 可以直接写出:

\begin{lstlisting}[language=c++,title={First},mathescape]
First($T$)={int, double}
First(${D}$)={, , $\varepsilon$}
First(${D}^{\prime }$)={int, double}
First(${L}^{\prime }$)={; , $\varepsilon$}
First($L$)={int, double}
\end{lstlisting}

\paragraph{Follow} 由于不存在隐式左递归,可以整理为:

\begin{itemize}
    \item $L \rightarrow  D{L}^{\prime }$
    \item ${L}^{\prime } \rightarrow  ;D{L}^{\prime } \mid  \varepsilon$
    \item $D \rightarrow  T\;{id\; D}^{\prime }$
    \item ${D}^{\prime } \rightarrow,id\;{D}^{\prime }$ | ${\varepsilon }$
    \item $T \rightarrow$ int | double
\end{itemize}

\begin{lstlisting}[language=c++,title={Follow},mathescape]
First($T$)={id, #}
First(${D}$)={; , #}
First(${D}^{\prime }$)={; , #}
First(${L}^{\prime }$)={#}
First($L$)={#}
\end{lstlisting}

\paragraph{LL(1) 分析表} 不难画出:



\begin{table}[htbp]
\centering
\renewcommand{\arraystretch}{1.2} % 增加行间距
\setlength{\tabcolsep}{10pt}      % 设置列间距
\caption{LL(1) 分析表}
\label{tab:LL1}
\begin{tabularx}{\textwidth}{>{\centering\arraybackslash}Xcccccc}
\toprule
 & , & ; & $id$ & int & double & $\#$ \\
\midrule
$L$ &  &  &  & $\#L \rightarrow DL^{\prime}$ & $\#L \rightarrow DL^{\prime}$ & \\
\rowcolor[gray]{.9} % 使用灰色背景以区分不同行
$L^{\prime}$ &  & $\#L^{\prime} \rightarrow ;DL^{\prime}$ &  &  &  & $\#L^{\prime} \rightarrow \varepsilon$\\
$D$ &  &  &  & $D \rightarrow T\;id\;D^\prime$ & $D \rightarrow T\;id\;D^\prime$ & \\
\rowcolor[gray]{.9}
$D^{\prime}$ & $D^{\prime} \rightarrow,id\;D^{\prime}$ & $D^{\prime} \rightarrow \varepsilon$ &  &  &  & $D^{\prime} \rightarrow \varepsilon$\\
$T$ &  &  &  & $T \rightarrow$ int & $T \rightarrow$ double & \\
\bottomrule
\end{tabularx}
\end{table}

\paragraph{int i; double x,y 分析过程:} 可以写出:


\begin{center}
\begin{tabular}{rl@{\hspace{10em}}r}
1 & $\#E$ & int $i$; double $x,y\#$ \\
\hline
2 & $\#L^\prime D^\prime$ & int $i$; double $x,y\#$ \\
\hline
3 & $\#L^\prime D^\prime\;id\;T$ & int $i$; double $x,y\#$ \\
\hline
4 & $\#L^\prime D^\prime\;id\;$ int & int $i$; double $x,y\#$ \\
\hline
5 & $\#L^\prime D^\prime\;id\;$ & $i$; double $x,y\#$ \\
\hline
6 & $\#L^\prime D^\prime$ & ; double $x,y\#$ \\
\hline
7 & $\#L^\prime$ & ; double $x,y\#$ \\
\hline
8 & $\#L^\prime D;$ & ; double $x,y\#$ \\
\hline
9 & $\#L^\prime D$ & double $x,y\#$ \\
\hline
10 & $\#L^\prime D^\prime\;id\;T$ & double $x,y\#$ \\
\hline
11 & $\#L^\prime D^\prime\;id$ double & double $x,y\#$ \\
\hline
12 & $\#L^\prime D^\prime\;id$ & $x,y\#$ \\
\hline
13 & $\#L^\prime D^\prime$ & $,y\#$ \\
\hline
14 & $\#L^\prime D^\prime\;id,$ & $,y\#$ \\
\hline
15 & $\#L^\prime D^\prime\;id$ & $y\#$ \\
\hline
16 & $\#L^\prime D^\prime$ & $\#$\\
\hline
17 & $\#L^\prime$ & $\#$\\
\hline
18 & $\#$ & $\#$\\
\end{tabular}
\end{center}