	% 代码分析:模块功能、涉及到的类、类关系、数据结构及关键代码等;
	% 任务要求,设计任务要求;
	% 设计:详细的设计方案,相关的数据结构、算法描述,可采用伪代码等形式化描述
	% 实现:修改哪些类、如何修改、为什么修改等;
	% 测试:测试用例,测试结果及结果分析,测试运行界面等;
	% 调试:调试方法,遇到的问题及解决方案等;
	% 结论与展望:完成的主要工作、收获、进一步的工作,建议、体会、心得等;

\section{作业7}
\subsection{作业7.1}
\subsubsection{题目描述}

本章中,关系式 $ {i}^{\left( 1\right) } < {i}^{\left( 2\right) } $ 被翻译成相继的两个四元式:

\begin{lstlisting}[language=c,title={${i}^{\left( 1\right) } < {i}^{\left( 2\right) }$},mathescape=true]
$\left( {j < ,{i}^{\left( 1\right) },{i}^{\left( 2\right) }, - }\right)$ // 真出口
$\left(j,-,-, -\right)$ // 假出口
\end{lstlisting}


这种翻译常常浪费一个四元式。如果我们翻译成如下四元式:

$ \left( {j \geq  ,{i}^{\left( 1\right) },{i}^{\left( 2\right) }, - }\right) $ // 假出口跳转,真出口自动滑到下一个四元式

那么,在 $ {i}^{\left( 1\right) } < {i}^{\left( 2\right) } $ 的情况下就不发生跳转(自动滑下来)。但若这个关系后有一个或运算,则另一个无条件转移指令是不可省的,例如:

\begin{lstlisting}[language=c,title={if $A < B \vee  C < D$ then $x = y$},mathescape=true]
100: $ \left( {j \geq  ,A,B,{102}}\right) $
101: $ \left( {j,-,-,{103}}\right) \; $ // 或运算前的无条件跳转不能省略
102: $\left( {j \geq  ,C,D,{104}}\right) $
103: $ \left( { = ,y,-,x}\right) $
\end{lstlisting}

请按上述要求改写翻译布尔表达式的语义动作。
\subsubsection{解答}
\begin{center}
\begin{longtable}{r|lll}
\toprule
E $\rightarrow$ A $\vee$ M E${}_1$    &  \{ & backpatch(A.falselist, M.quad); & \\
& &    E.truelist = merge(A.truelist, E${}_1$.truelist); & \\
& &    E.falselist = E${}_1$.falselist; & \} \\
\hline
E $\rightarrow$ E${}_1$ $\wedge$ M E${}_2$  &    \{ & backpatch(E.truelist, M.quad); & \\
& &    A.falselist = merge(E${}_1$.falselist, E${}_2$.falselist); & \\
& &    E.truelist = E${}_2$.truelist; & \} \\
\hline
M $\rightarrow \varepsilon$       &    \{ & M.quad = nextquad; & \} \\
\hline
E $\rightarrow$ 7 E      &     \{ & E.truelist = E${}_1$.falselist & \\
& &    E.falselist = E${}_1$.falselist & \} \\
\hline
E $\rightarrow$ (E${}_1$)      &     \{ & E.truelist = E${}_1$.truelist & \\
& &    E.falselist = E${}_1$.falselist & \} \\
\hline
A $\rightarrow$ A${}_1$ $\vee$ M A${}_2$    &  \{ & backpatch(A.falselist, M.quad); & \\
& &    A.truelist = merge(A${}_1$.truelist, A${}_2$.truelist); & \\
& &    A${}_1$.falselist = A${}_2$.falselist; & \} \\
\hline
A $\rightarrow$ E $\wedge$ M A${}_1$  &    \{ & backpatch(E.truelist, M.quad); & \\
& &    A.falselist = merge(E.falselist, A${}_1$.falselist); & \} \\
\hline
A $\rightarrow$ 7 A${}_1$      &     \{ & A.truelist = A${}_1$.truelist; & \\
& &    A.falselist = A${}_1$.falselist; & \} \\
\hline
A $\rightarrow$ (A)      &     \{ & A.truelist = A${}_1$.truelist; & \\
& &    A.falselist = A${}_1$.falselist; & \} \\
\hline
E $\rightarrow$ id${}_1$ $\theta$ id${}_2$  & \{ & E.truelist = makelist(); & \\
& &    E.falselist = makelist(nextquad); & \\
& &    gen(~j$\theta$, id${}_1$.name, id${}_2$.name, 0); & \} \\
\hline
A $\rightarrow$ id${}_1$ $\theta$ id${}_2$ &  \{ & A.truelist = makelist(nextquad + 1); & \\
& &    A.falselist = makelist(nextquad); & \\
& &    gen(~j$\theta$, id${}_1$.name, id${}_2$.name, 0); & \\
& &    gen(j, -, -, 0); & \} \\
\bottomrule
\end{longtable}
\end{center}
\subsection{作业7.2}
\subsubsection{题目描述}
根据本章所述翻译模式,将如下声明语句填符号表(名字、类型、偏移量三项),其他语句翻译为四元式。
\begin{lstlisting}[language=c,title={四元式翻译}]
int i, j;
real x, y;
i = 1;
j = 1;
while (i <= 9)
{
	while (j <= 9)
	{
		a[i, j] = i * j;
		j = j + 1;
	}
	i = i + 1;
}
\end{lstlisting}

\subsubsection{解答}

可以写出符号表:


\begin{table}[H]
    \centering
    \begin{tabular}{rll}
    \hline
i  &  int   &  0 \\
j  &  int   &  4 \\
x  &  real  &  8 \\
y  &  real  &  12 \\
a  &  array &  16 \\
\hline
    \end{tabular}
    \caption{符号表}
    \label{tab:symbol}
\end{table}

不难翻译出:

\begin{lstlisting}[language=c,title={四元式翻译}]
100: (=,1,-,T0_i)
101: (=,T0_i,-,TB0)
102: (=,1,-,T1_i)
103: (=,T1_i,-,TB1)
104: (=,9,-,T2_i)
105: (j<=,TB0,T2_i,107)
106: (j,-,-,120)
107: (=,9,-,T3_i)
108: (j<=,TB1,T3_i,110)
109: (j,-,-,116)
110: (*,TB0,TB1,T4_i)
111: (=,T4_i,-,TB2)
112: (=,1,-,T5_i)
113: (+,TB1,T5_i,T6_i)
114: (=,T6_i,-,TB1)
115: (j,-,-,107)
116: (=,1,-,T7_i)
117: (+,TB0,T7_i,T8_i)
118: (=,T8_i,-,TB0)
119: (j,-,-,104)
120: (End,-,-,-)
\end{lstlisting}