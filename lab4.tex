	% 代码分析:模块功能、涉及到的类、类关系、数据结构及关键代码等;
	% 任务要求,设计任务要求;
	% 设计:详细的设计方案,相关的数据结构、算法描述,可采用伪代码等形式化描述
	% 实现:修改哪些类、如何修改、为什么修改等;
	% 测试:测试用例,测试结果及结果分析,测试运行界面等;
	% 调试:调试方法,遇到的问题及解决方案等;
	% 结论与展望:完成的主要工作、收获、进一步的工作,建议、体会、心得等;

\section{作业4}
\subsection{作业4.1}
\subsubsection{题目描述}

将右线性文法$G[S]: S \rightarrow xA\;|\;yB\;|\;\varepsilon\;,\;A \rightarrow yA\;|\;y\;,\;B \rightarrow xB\;|\;x$, 转换为:
\begin{enumerate}
    \item 有限自动机。
    \item 正则式。
\end{enumerate}

\subsubsection{解答}

\paragraph{有限自动机。} 令有限自动机 $M=\left(\{S,A,B,f\},\{x,y\},\delta,\{S\},\{f\}\right)$,其中 $f$ 为添加的终态符号。

类似例题 3.15 的步骤,构造 $\delta$:

\begin{enumerate}
    \item 对于产生式 $S \rightarrow xA$,由 $S$ 向 $A$ 引 $x$ 弧。
    \item 对于产生式 $S \rightarrow yB$,由 $S$ 向 $B$ 引 $y$ 弧。
    \item 对于产生式 $S \rightarrow \varepsilon$,由 $S$ 向 $f$ 引 $\varepsilon$ 弧。
    \item ...
\end{enumerate}

不难画出:

\begin{center}
\begin{tikzpicture}[>=stealth',shorten >=1pt,auto,node distance=2cm]
  \node[initial,state] (S)      {$S$};
  \node[state]         (A) [right of=S]  {$A$};
  \node[state]         (B) [below of=S] {$B$};
  \node[state,accepting](f) [below of=A] {$f$};

  \path[->]  (S)  edge node {$x$} (A);
  \path[->]  (S)  edge node {$y$} (B);
  \path[->]  (S)  edge node {$\varepsilon$} (f);
  \path[->]  (A)  edge[loop right] node {$y$} (A);
  \path[->]  (B)  edge[loop left] node {$x$} (B);
  \path[->]  (A)  edge node {$y$} (f);
  \path[->]  (B)  edge node {$x$} (f);
\end{tikzpicture}
\end{center}

\paragraph{正规式} 正规式可以直接与线性文法转换,于是就不借助有限自动机了(虽然这个题使用自动机转换也很简单)。

首先改写:

\begin{enumerate}
    \item $S \rightarrow xA\;|\;yB\;|\;\varepsilon\;$ 不包含本身,直接写为:$S \rightarrow (xA|yB|\varepsilon)$;
    \item $A \rightarrow yA\;|\;y\;$ 改写为 $A \rightarrow (yA|y)$,也就是 $A\rightarrow y^*y$,$B$ 完全同理,改写为 $B\rightarrow x^*x$;
    \item 将第二步的结果代入第一步,得到 $S \rightarrow (xy^*y|yx^*x|\varepsilon)$。
\end{enumerate}

最终得到正规式为 $(xy^*y|yx^*x|\varepsilon)$,如果允许使用 $+$ 与 $?$,可以简写为 $(xy^+|yx^+)?$。

\subsection{作业4.2}
\subsubsection{题目描述}
给定右线性文法$G[S]$, 求其等价的左线性文法:
\begin{align*}
    S&\rightarrow 0S\;|\;1S\;|\;1A\;|\;0B\\
    A&\rightarrow 1C\;|\;1\\
    B&\rightarrow 0C\;|\;0\\
    C&\rightarrow 0C\;|\;1C\;|\;0\;|\;1
\end{align*}
\subsubsection{解答}

字符数较少,似乎用正则比较方便?,下面通过右线性文法转正规式,再将正规式转左线性文法来实现转换:

\paragraph{转正规式} 与上题类似,首先改写:(下方便起见,记$\Sigma=\{0,1\}$)

\begin{itemize}
    \item $S\rightarrow 0S\;|\;1S\;|\;1A\;|\;0B$ 首先写为 $S\rightarrow (\Sigma S|1A|0B)$,也就是 $S\rightarrow \Sigma^*(1A|0B)$;
    \item $A\rightarrow 1C\;|\;1$ 简写为 $A\rightarrow 1C?$,$B$ 同理简写为 $0C?$;
    \item $C\rightarrow 0C\;|\;1C\;|\;0\;|\;1$ 首先写为 $C\rightarrow (\Sigma C|\Sigma)$,也就是 $\Sigma^+$;
    \item 依次代入上面式子,得到 $S\rightarrow \Sigma^*(11|00)\Sigma^+?$
\end{itemize}

得到正规式为 $(0|1)^*(11|00)(((1|0)(1|0)^*)|\varepsilon)$。

\paragraph{左线性文法} 类似上文,进行逐步拆解:

\begin{enumerate}
    \item ${\color{Red} \Sigma^*} ({\color{Purple} 11} |{\color{Orange} 00} ){\color{Green} \Sigma^+ ?} $ 替换为 $S\rightarrow S_1 {\color{Green} \Sigma^+ ?}$、$S_1\rightarrow S_2{\color{Purple} 1}|S_3{\color{Orange} 0}$、$S_2\rightarrow S_4{\color{Purple} 1}$、$S_3\rightarrow S_4{\color{Orange} 0}$、$S_4\rightarrow {\color{Red} \Sigma^*}$;
    \item $S\rightarrow S_1 \Sigma^+?$ 可写为 $S\rightarrow S_1 0\;|\;S_11\;|\;S0\;|\;S1$;
    \item $S_4\rightarrow \Sigma^*$ 可写为 $S_4\rightarrow S_40\;|\;S_41\;|\;0\;|\;1\;|\;\varepsilon$;
\end{enumerate}

总伤,该正规式 ${\color{Red} \Sigma^*} ({\color{Purple} 11} |{\color{Orange} 00} ){\color{Green} \Sigma^+ ?} $,也就是给出的右线性文法对应的左线性文法为:

\begin{align*}
    S&\rightarrow {\color{Purple} S_21} \;|\;{\color{Orange} S_30} \;|\;{\color{Green} S_1 0\;|\;S_11\;|\;S0\;|\;S1} \\
    S_1&\rightarrow {\color{Purple} S_21} \;|\;{\color{Orange} S_30} \\
    S_2&\rightarrow {\color{Purple} S_41} \\
    S_3&\rightarrow {\color{Orange} S_40} \\
    S_4&\rightarrow {\color{Red} S_40\;|\;S_41\;|\;0\;|\;1\;|\;\varepsilon} 
\end{align*}